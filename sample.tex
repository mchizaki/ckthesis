\documentclass[
	english,
	times,
	fontsize = 11pt,
	left     = 35truemm,
	right    = 20truemm,
	top      = 15truemm,
	bottom   = 20truemm,
	headrule,
	% draft,
	% tombo,
	% headfontsc,
	% headsep  = 10truemm,
	fleqn,
	reversemarginpar
]{ckthesis}


%%%%%%%%%%%%%%%%%%%%%%%%%%%%%%%%%%%%%%%%%%%%%%%%%%%%%%%%%%%%%%%%%%
% preamble
%%%%%%%%%%%%%%%%%%%%%%%%%%%%%%%%%%%%%%%%%%%%%%%%%%%%%%%%%%%%%%%%%%
% figure
% \graphicspath{%
% 	...
% }

%===============================%
% math font
%===============================%
% \usepackage[
% 	subscriptcorrection,
% 	slantedGreek,
% 	nofontinfo,
% 	mtpcal
% ]{mtpro2}


%===============================%
% math packages
%===============================%
% math
\usepackage{amsmath}
\usepackage{mathtools}
\usepackage{empheq}
\usepackage{bm}
\usepackage{siunitx}
\usepackage{physics}
\usepackage{braket} % after reading physics package


%===============================%
% todo
%===============================%
\usepackage[
        textsize=scriptsize,
        % textsize=footnotesize,
        % color=Magenta,
        linecolor=Magenta,
        bordercolor=Magenta,
        backgroundcolor=White,
        disable
        % textcolor
]{todonotes}
\usepackage{soul}
\makeatletter
	\if@todonotes@disabled
	\newcommand{\hlnote}[3][0pt]{#2}
	\else
	\newcommand{\hlnote}[3][0pt]{\texthl{#2}\vspace{-#1}\todo{#3}\vspace{#1}}
	\fi
\makeatother


%===============================%
% caption
%===============================%
\usepackage{caption}
\captionsetup[figure]{
	textfont={small},
	labelfont={sf,bf,small},
	margin=0pt,
	labelsep=space,
	name=Figure\ {}%
}
\captionsetup[table]{
	textfont={small},
	labelfont={sf,bf,small},
	margin=0pt,
	labelsep=space,
	name=Table\ {}%
}


%===============================%
% other packages
%===============================%
\usepackage{xspace}
\usepackage{here}
\usepackage{enumitem}
\usepackage[cmyk,dvipsnames]{xcolor}
\usepackage[bookmarks]{hyperref}
\usepackage[
	style=phys,
	backend=biber,
	sorting=none
]{biblatex}
\usepackage[utf8]{inputenc}
\usepackage[figure,table]{totalcount}


\ExecuteBibliographyOptions{%
	date=year,
	isbn=false,
	url=false,
	doi=false,
	giveninits=true,
	biblabel=brackets
}

\hypersetup{%
	bookmarksnumbered=true,
	colorlinks = true,
	allcolors  = NavyBlue,
	unicode    = true
}


%===============================%
% Reference file
%===============================%
\addbibresource{reference.bib}


%===============================%
% document information
%===============================%
\title{%
	Thesis title
}

\author{%
	Author name
}
\date{%
	date
}

\University{%
	University
}
\Department{%
	Department
}
\SideTextAboveTitle{%
	SideTextAboveTitle
}
\CenterTextAboveTitle{%
	CenterTextAboveTitle
}





\begin{document}
\pagestyle{empty}


%%%%%%%%%%%%%%%%%%%%%%%%%%%%%%%%%%%%%%%%%%%%%%%%%%%%%%%%%%%%%%%%%%
% document (front)
%%%%%%%%%%%%%%%%%%%%%%%%%%%%%%%%%%%%%%%%%%%%%%%%%%%%%%%%%%%%%%%%%%
\frontmatter
\maketitle


%-------------------------------%
% Abstract
%-------------------------------%
\pagestyle{plain}
\cleardoublepage

% Abstract(Japanese)
\ckCenterTitle{概要}
\addcontentsline{toc}{chapter}{概要}
この文書クラス"ckthesis"はltjsbookを元に作成したものです。
日本語と英語の両方に対応しています。

\clearpage


% Abstract(English)
\ckCenterTitle{Abstract}
\addcontentsline{toc}{chapter}{Abstract}
Lorem ipsum dolor sit amet, consectetur adipiscing elit,
sed do eiusmod tempor incididunt ut labore et dolore magna aliqua.
Ut enim ad minim veniam,
quis nostrud exercitation ullamco laboris nisi ut aliquip ex ea commodo consequat.
Duis aute irure dolor in reprehenderit in voluptate velit esse cillum dolore eu fugiat nulla pariatur.
Excepteur sint occaecat cupidatat non proident,
sunt in culpa qui officia deserunt mollit anim id est laborum.


%-------------------------------%
% Table of contents
%-------------------------------%
{
	\hypersetup{linkcolor=black}

	% Table of Contents
	\tableofcontents
	\addcontentsline{toc}{chapter}{\contentsname}

	% List of Figures
	\iftotalfigures
		\listoffigures
		\addcontentsline{toc}{chapter}{\listfigurename}
	\fi

	% List of Tables
	\iftotaltables
		\listoftables
		\addcontentsline{toc}{chapter}{\listtablename}
	\fi
}




%%%%%%%%%%%%%%%%%%%%%%%%%%%%%%%%%%%%%%%%%%%%%%%%%%%%%%%%%%%%%%%%%%
% document (main)
%%%%%%%%%%%%%%%%%%%%%%%%%%%%%%%%%%%%%%%%%%%%%%%%%%%%%%%%%%%%%%%%%%
\mainmatter
\pagestyle{headings}

% Introduction
% You can separate TeX files by using \input or \include
% \chapter{序章}
% 山路を登りながら,こう考えた。
% 智に働けば角が立つ。
% 情に棹させば流される。
% 意地を通せば窮屈だ。
% とかくに人の世は住みにくい。
% 住みにくさが高じると,安い所へ引き越したくなる。
% どこへ越しても住みにくいと悟った時,詩が生れて,画が出来る。
% \begin{equation}
%     \qty( \int_{0}^{\infty} \frac{\sin x}{\sqrt{x}} \dd x )^2
%     =
%     \sum_{ k = 0 }^{ \infty }
%         \frac{ (2k)! }{ 2^{2k} (k!)^2 } \frac{1}{ 2k+1 }
%     =
%     \prod_{ k = 1 }^{ \infty }
%         \frac{ 4k^2 }{ 4k^2 - 1 }
%     = \frac{ \pi }{ 2 }
% \end{equation}




\chapter{Introduction}
Introduction

\section{Section 1}
\subsection{Subsection 1}
Lorem ipsum dolor sit amet, consectetur adipiscing elit,
sed do eiusmod tempor incididunt ut labore et dolore magna aliqua.
Ut enim ad minim veniam,
quis nostrud exercitation ullamco laboris nisi ut aliquip ex ea commodo consequat.
Duis aute irure dolor in reprehenderit in voluptate velit esse cillum dolore eu fugiat nulla pariatur.
Excepteur sint occaecat cupidatat non proident,
sunt in culpa qui officia deserunt mollit anim id est laborum\cite{PaperSample1}.
\begin{equation}
\qty( \int_{0}^{\infty} \frac{\sin x}{\sqrt{x}} \dd x )^2
    =
    \sum_{ k = 0 }^{ \infty }
        \frac{ (2k)! }{ 2^{2k} (k!)^2 } \frac{1}{ 2k+1 }
    =
    \prod_{ k = 1 }^{ \infty }
        \frac{ 4k^2 }{ 4k^2 - 1 }
    = \frac{ \pi }{ 2 }
\end{equation}

% \begin{figure}
%     \centering
%     \includegraphics[options]{name}
%     \caption{caption}
%     \label{fig:1}
% \end{figure}


\clearpage


\section{Section 2}
Lorem ipsum dolor sit amet, consectetur adipiscing elit,
sed do eiusmod tempor incididunt ut labore et dolore magna aliqua.
Ut enim ad minim veniam,
quis nostrud exercitation ullamco laboris nisi ut aliquip ex ea commodo consequat.
Duis aute irure dolor in reprehenderit in voluptate velit esse cillum dolore eu fugiat nulla pariatur.
Excepteur sint occaecat cupidatat non proident,
sunt in culpa qui officia deserunt mollit anim id est laborum\cites{PaperSample1,PaperSample2,PaperSample3}.




% Main contents
\chapter{Main}
Main


% Conclusion
\chapter*{Conclusion}
\addcontentsline{toc}{chapter}{Conclusion}
Conclusion


%-------------------------------%
% Appendix
%-------------------------------%
\appendix
% Appendix A
\chapter{Supplementary information 1}
Appendix A


% Appendix B
\chapter{Supplementary information 2}
Appendix B


%-------------------------------%
% Reference
%-------------------------------%
\printbibliography[title=References]
\addcontentsline{toc}{chapter}{References}




%%%%%%%%%%%%%%%%%%%%%%%%%%%%%%%%%%%%%%%%%%%%%%%%%%%%%%%%%%%%%%%%%%
% document (back)
%%%%%%%%%%%%%%%%%%%%%%%%%%%%%%%%%%%%%%%%%%%%%%%%%%%%%%%%%%%%%%%%%%
\backmatter

% Acknowledgments
\chapter*{Acknowledgments}
\addcontentsline{toc}{chapter}{Acknowledgments}
Acknowledgments


% Research achievements
\chapter*{Research achievements}
\addcontentsline{toc}{chapter}{Research achievements}
Research achievements



\end{document}
